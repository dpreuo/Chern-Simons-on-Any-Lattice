\appendix
\section{Proof of the Simplicial Product Rule} \label{apx:chain_proof}

These proofs are kind of sticky, so have been relegated to an appendix.
Let us start by restating the four identities we need,
\begin{align}
    \partial \gamma &=\sum_{[v_1, ... ,v_{n}])} (v_1, ... ,v_{n}) 
    \sum_{[v_0]\rightarrow [v_0, ... ,v_{n}]}
    \gamma_{v_0, v_1, ... ,v_{n}}
    \\
    d\gamma
    &= \sum_{[v_0, ... , v_{n+1}]}
    (v_0, ... , v_{n+1})
    \sum_{m = 0}^{n+1}(-1)^m
    \gamma_{v_0...\hat v_m...v_{n+1}},
    \\
    \alpha^n \smile \beta^m &= \sum_{[v_0, ...,v_{n+m}]} 
    \sum_{P} \frac{\sgn(P)}{(n+m+1)!} \alpha_{p_0 ...p_n}\beta_{p_n ...p_{n+m}} (v_0, ...v_n,...,v_{n+m}),
    \\
    \alpha^n \frown \beta^m &= \sum_{[v_0, ...,v_{n}]} 
    \sum_{P} \frac{\sgn(P)}{(n+1)!} \alpha_{p_0 ...p_n}\beta_{v_0 ...v_m} (v_m,...,v_{n}).
\end{align}


\subsection{The Cup Product}
To do so, let us explicitly calculate each of the three expressions in this identity
\begin{align}
    d (\alpha^n \smile \beta^m) &=
    \sum_{[v_0...v_{n+m+1}]} 
    (v_0 ... v_{n+m+1})
    \sum_{u = 0}^{n+m+1}(-1)^u
    \sum_{P_{\hat u}} 
    \frac{\sgn(P_{\hat u})}{(n+m+1)!}
    \alpha_{p_0...p_n} 
    \beta_{p_n...p_{n+m}}
    \\
    d \alpha^n \smile \beta^m &= 
    \sum_{[v_0...v_{n+m+1}]}
    (v_0 ... v_{n+m+1})
    \sum_{P}\frac{\sgn(P)}{(n+m+2)!}
    \sum_{u = 0}^{n}
    \\
    (-1)^n \alpha^n \smile d\beta^m &= 
    \sum_{[v_0...v_{n+m+1}]},
\end{align}
where $P_{\hat u}$ denotes a permutation of the indices $(v_0 ... \hat v_u ... v_{n+m+1})$, where vertex $v_u$ has been removed.

\subsection{The Cap Product}

\subsection{Some Examples}
\begin{shaded}
    {\it Single mom finds one weird vector identity for simplifying equations FAST --- } There is only one vector identity that is really relevant to this action, and it is the same one we used in \textsection{\ref{sec:continuous_special_time}}:
    \begin{align} 
            \nabla \times (a \textbf{A}) = \nabla a \times \textbf{A} + a \nabla \times \textbf{A},
        \end{align}
        where $a$ is a scalar and $\textbf A$ is a 1-field. Let us derive the simplicial equivalent of this expression
        \begin{align}
            d(\phi \alpha) = \sum_{[i,j]}\frac 12 (\phi_i + \phi_j) \alpha_{ij} d(i,j),
        \end{align}
        where we have used the definition presented above for 0-1-form multiplication. Taking the coboundary of (i,j) we arrive at the following
        \begin{align}
             d(\phi \alpha) &= \sum_{[i,j,k]}\frac 12 
             \left [
             (\phi_i + \phi_j)\alpha_{ij}
             + (\phi_j + \phi_k)\alpha_{jk}
             +(\phi_k + \phi_i)\alpha_{ki}
             \right ]
        \end{align}
    which is rearranged to the form
    \begin{align}
        \begin{aligned}
            d(\phi \alpha) &= \sum_{[i,j,k]} \frac 13 (\phi_i + \phi_j +\phi_k)(\alpha_{ij} + \alpha_{jk}+\alpha_{ki}) \\ 
            &
            + \sum_{[i,j,k]} \frac 16 \bigg [
            [(\phi_i - \phi_k) + (\phi_j - \phi_k)]\alpha_{ij}
            + [(\phi_j - \phi_i) + (\phi_k - \phi_i)]\alpha_{jk}\\& \qquad \qquad 
            + [(\phi_k - \phi_j) + (\phi_i - \phi_j)]\alpha_{ki}
            \bigg ] (i,j,k)
        \end{aligned}
    \end{align}
    This looks like a mess, but consulting our earlier identities for cross and scalar product, we can actually see that this exactly evaluates
    \begin{align}
        d(\phi \alpha) = \phi d\alpha + d\phi \times \alpha.
    \end{align}
\end{shaded}